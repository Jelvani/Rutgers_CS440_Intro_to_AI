\section{Introduction}
\label{sec:Introduction}

In this project, we will first perform an analysis of the classical $A^* search$  algorithm. To begin, it is important to define our environment and testing conditions. All tests are performed on a 101 X 101 grid where each cell is considered an atomic unit of the map. The map world is defined to be \emph{static} and initially \emph{unknown}, meaning the cells that are walls do not change as the \emph{agent} moves and the location of the walls is initially unknown. The agent can start at any cell for its \emph{initial state} and the agent is to navigate the \emph{partially observable} environment. That is, the agents \emph{actions} include checking any adjacent cells for walls and moving to any adjacent cells.


Now We will briefly explain our implementation details with Python that are required for this problem. Our \emph{state space} consists of all reachable cells from the initial state, and we will therefore need a way to store each state along with the path taken to reach that state. We provide a \texttt{Node} class that looks like the following:
\begin{figure}[h]
\begin{lstlisting}
class Node():
    def __init__(self,parent = None,position = None,g = 0,h = 0):
        self.parent = parent #type Node
        self.position = position #type tuple
        self.g = g
        self.h = h
    def __eq__(self, otherNode): #for '==' comparison of a Node object
        return otherNode.position == self.position
    def __lt__(self, otherNode): #for '<' comparison of a Node object
        if self.f == otherNode.f:
            return self.g >= otherNode.g #tie breaker for same f value
        else:
            return self.f < otherNode.f
    @property
    def f(self):
        return self.g + self.h
\end{lstlisting}
\caption{A Python implementation of the \texttt{Node} class, which is created for each cell in our open list.}
\end{figure}


It is important to note our parent variable, which is type \texttt{Node}. This will allow us to create a linked list that represents the path taken to reach the current node. Furthermore, we will use the \texttt{Numpy} library for creating our binary 101 X 101 matrix that will represent our map with walls. The \texttt{Numpy} library also lets us easily store our generated mazes for future use. It is also important to have a efficient data structure for the priority queue required by $A^* search$. We will discuss $A^*$ in more depth in later sections, but since we need to explore the node with the smallest evaluation function, given by $f(n) = g(n) + h(n)$, we do not want to iterate over all values in the open list containing the nodes. Therefore, we use the \texttt{heapq} library, which implements a binary heap. This is also why we provide the \texttt{lt} definition in our \texttt{Node} class, since it is used by Python has an overloading operator for the \emph{less than} operator so the \texttt{heapq} library can maintain the binary heap data structure. Compared to a normal linear search with a worst case run time of $\mathcal{O}(n)$, a binary heap will give us the lowest evaluation function valued node in $\mathcal{O}(\log n)$ worst case run time.

For visualization purposes we will use the \texttt{matplotlib} library. This allows us to view our \texttt{numpy} arrays on a graph and will also help us save the generated graphs to an image file.



\emph{** Update: For this paper, figures displayed were using }\texttt{matplotlib}. \emph{In the latest version of the repository, the} \texttt{pygame} \emph{library is used.}